\documentclass{article}

\usepackage{ucs}

\usepackage[utf8x]{inputenc}
\usepackage[greek, english]{babel}
\usepackage{alphabeta}
\usepackage{lmodern}

\usepackage[linguistics]{forest}

\usepackage{listings}

\usepackage{graphicx}
\graphicspath{./images/}

\usepackage{forest}

\title{Τρίτη εργαστρηριακή εργασία στο μάθημα Εισαγωγή στον Παράλληλο Υπολογισμό}
\date{2021-01-02}
\author{Ιάκωβος Μαστρογιαννόπουλος - cse242017102}

\definecolor{codegreen}{rgb}{0,0.6,0}
\definecolor{codegray}{rgb}{0.5,0.5,0.5}
\definecolor{codepurple}{rgb}{0.58,0,0.82}
\definecolor{backcolour}{rgb}{0.95,0.95,0.95}

\lstdefinestyle{mystyle} {
    backgroundcolor=\color{backcolour},
    commentstyle=\color{codegreen},
    keywordstyle=\color{magenta},
    numberstyle=\tiny\color{codegray},
    stringstyle=\color{codepurple},
    basicstyle=\ttfamily\footnotesize,
    breakatwhitespace=false,
    breaklines=true,
    captionpos=b,
    keepspaces=true,
    numbers=left,
    numbersep=5pt,
    showspaces=false,
    showstringspaces=false,
    showtabs=false,
    tabsize=2
}

\lstset{style=mystyle}

\begin{document}
    \pagenumbering{gobble}
    \maketitle

    \newpage
    \tableofcontents

    \newpage
    \begin{abstract}
        Αυτή είναι η τρίτη εργαστηριακή εργασία στο μάθημα Εισαγωγή στον Παραλλήλο Υπολογισμού του 5ου εξαμήνου.
    \end{abstract}

    \newpage
    \section{Κατανόηση}
    \paragraph{}
    Το πρόβλημα που μας ζητείται να επιλύσουμε το παρακάτω ζητούμενο: έχουμε μια καρτησιανή τοπολογία ΜxN, και πρέπει
    με την χρήση της C και του Framework MPI να βρούμε το καρτησιανό γινόμενο. Μας δώθηκε η επιλόγη να επιλέξουμε μεταξύ 
    \texttt{MPI\_Send()} και \texttt{MPI\_Recv()} ή με \texttt{MPI\_Cart\_sub()} και \texttt{MPI\_Reduce()}. 
    Επιλέχτηκε να υλοποιηθεί με το πρώτο τρόπο. Τα δεδομένα μπορεί να τα δίνει ο χρήστης τα δεδομένα από το πληκτρολόγιο ή
    από αρχείο.

    \section{Υλοποίηση}
    \lstinputlisting[language=C]{ParComp3.c}

\end{document}